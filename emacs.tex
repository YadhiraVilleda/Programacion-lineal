\documentclass{article}
\usepackage[spanish]{babel}

\usepackage{amsmath}
\usepackage[utf8]{inputenc}

\title{MÉTODO SIMPLEX}
\author{Yadhira Villeda}

\begin{document}
\maketitle
\section{Introducción}
\label{sec:introduccion}

El metodo simplex es un algoritmo para resolver problemas de
programacion lineal.Fue inventado por George Datzig en el año 1947.

\section{Ejemplo}
\label{sec:ejemplo}
Ilustraremos la aplicación del método simplex con  un ejemplo:

\begin{equation*}
  \begin{aligned}
    \text{Maximizar} \quad &2x+2y\\
    \text{sujeto a}\quad &
    \begin{aligned}
      2x+y &\leq 4\\
      x+2y &\leq 5 \\
       x,y &\geq 0
    \end{aligned}
  \end{aligned}
\end{equation*}
Se convierte a el siguiente problema, agregando las variables de
holgura:\\

\begin{equation*}
  \begin{aligned}
    \text{Maximizar} \quad &2x_1+2x_2\\
    \text{sujeto a}\quad &
    \begin{aligned}
      2x_1+x_2+x_3 = 4\\
      x_1+2x_2+x_4= 5 \\
       x_1,x_2,x_4,x_3 &\geq 0
    \end{aligned}
  \end{aligned}
\end{equation*}
A continuación obtenemos un\emph{Tablero Simplex} despejando las variables
de holgura quedando:\\

\begin{equation*}
   \begin{aligned}
     x_3 &=-2x_1-x_2 +4\\
     x_4 &=-x_1-2x_2+ 5 \\
     \hline
     z &=2x_1+2x_2
    \end{aligned}
\end{equation*}

\end{document}

