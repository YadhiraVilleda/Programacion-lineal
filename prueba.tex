\documentclass{article}

\title{Apuntes de programaci\'on lineal}
\author{Yadhira Villeda Trejo}
\usepackage[utf8]{inputenc}
\usepackage{amsmath}


\begin{document}
\maketitle
\section{Forma Éstandar}
La forma estándar de un problema de programacón lineal es:
Dada una matriz $A$ y vectores $c.b$, maximizar $c^Tx$ sujeto a:\\  
$Ax\leq b$ \\
$x\leq 0$

\subsection{Ejemplos}
\label{sec:ejemplos}

Maximizar $x+y$ \\
sujeto a:\\
$y-x\leq 1$ \\
$x+6y\leq 15$ \\
$4x-y\leq 10$ \\
la forma estandar es:\\

$c=(1,1)$
\begin{equation}
  \label{eq:1}
  A=\begin{pmatrix}
    1&-1\\
    1&6 \\
    4&-1
  \end{pmatrix},
  c= \begin{pmatrix}
    1\\
    1
  \end{pmatrix}, b=
  \begin{pmatrix}
    1\\
    15\\
    10
  \end{pmatrix}


\end{equation}
\section{Forma Simplex}
La forma simplex de un probema de programacion lineal es: sea la
matriz $A$ y los vectores $c,b$ maximizar $c^Tx$
sujeto a:\\
$Ax=b$ \\
$x\leq 0$








\end{document}